\documentclass[twoside,11pt]{homework}

\coursename{COMS 4772 Fall 2015} % DON'T CHANGE THIS

\studentname{Name Surname}       % YOUR NAME GOES HERE
\studentmail{uni@columbia.edu}   % YOUR UNI GOES HERE
\homeworknumber{1}               % THE HOMEWORK NUMBER GOES HERE
\collaborators{none}             % THE UNI'S OF STUDENTS YOU DISCUSSED WITH

\begin{document}
\maketitle

\section*{Problem 1}

% YOUR SOLUTION GOES HERE

% SOME EXAMPLE LATEX CODE BELOW (DON'T INCLUDE IN YOUR ACTUAL SUBMISSION!)
\textbf{Examples of blackboard and calligraphic letters}: $\bbR^d
\supset \bbS^{d-1}$, $\cC \subset \cB$.
We usually reserve $\bbR$ for the real numbers, $\bbN$ for the natural
numbers, $\bbZ$ for the integers, etc.
These are defined through macros
\texttt{$\backslash$bbR},
\texttt{$\backslash$bbS},
\texttt{$\backslash$cC},
\texttt{$\backslash$cB},
etc.

\textbf{Examples of bold-faced letters}, perhaps suitable for matrix
and vectors:
\begin{equation}
  L(\bfx,\bflambda) = f(\bfx) - \dotp{\bflambda,\bfA\bfx-\bfb} .
  \label{eq:lagrangian}
\end{equation}
These are defined through macros
\texttt{$\backslash$bfx},
\texttt{$\backslash$bflambda},
\texttt{$\backslash$bfA},
\texttt{$\backslash$bfb},
etc.
The inner product uses the
\texttt{$\backslash$dotp}
macro.

\textbf{Example of a math operator}:
\[
  \var(X) = \bbE X^2 - (\bbE X)^2 .
\]
The
\texttt{$\backslash$var}
macro is defined using
\texttt{$\backslash$DeclareMathOperator}.

\textbf{Example of references}: the Lagrangian is given in
Eq.~\eqref{eq:lagrangian}, and Theorem~\ref{thm:euclid} is
interesting.
If the references show up as question marks, check that the reference
is valid, and then also just try running the \LaTeX compiler once or
twice more.

\textbf{Example of adaptively-sized parentheses}:
using the \texttt{$\backslash$Parens} macro,
\[
  \left(\prod_{i=1}^n x_i\right)^{1/n}
  + \left(\prod_{i=1}^n y_i\right)^{1/n}
  \leq
  \Parens{
    \prod_{i=1}^n (x_i + y_i)
  }^{1/n}
\]
(also have macros for \texttt{$\backslash$Braces},
\texttt{$\backslash$Brackets}, \texttt{$\backslash$Norm}, etc.).

\textbf{Example of aligned equations}:
\begin{align}
  \Pr(X = 1 \,|\, Y = 1)
  & = \frac{\Pr(X = 1 \,\wedge\, Y = 1)}{\Pr(Y = 1)}
  \notag \\
  & =
  \underbrace{
    \frac{\Pr(Y = 1 \,|\, X = 1) \cdot \Pr(X = 1)}{\Pr(Y = 1)}
  }_{\text{Usual expression for Bayes' rule}}
  .
  \label{eq:bayes-rule}
\end{align}

\textbf{Example of a theorem}:
\begin{theorem}[Euclid]
  \label{thm:euclid}
  There are infinitely many primes.
\end{theorem}
\begin{proof}[Euclid's proof]
  There is at least one prime, namely $2$.
  Now pick any finite list of primes $p_1, p_2, \dotsc, p_n$.
  It suffices to show that there is another prime not on the list.
  Let $p := \prod_{i=1}^n p_i + 1$, which is not any of the primes on
  the list.
  If $p$ is prime, then we're done.
  So suppose instead that $p$ is not prime.
  Then there is prime $q$ which divides $p$.
  If $q$ is one of the primes on the list, then it would divide $p -
  \prod_{i=1}^n p_i = 1$, which is impossible.
  Therefore $q$ is not one of the $n$ primes in the list, so we're
  done.
\end{proof}

\textbf{Here is a centered table}:
\begin{center}
  \begin{tabular}{c||c|c|c|c}
    & A
    & B
    & C
    & D \\
    \hline
    \hline
    $1$
    & entries
    & in
    & a
    & table
    \\
    \hline
    $2$
    & more
    & entries
    & more
    & entries
  \end{tabular}
\end{center}

\textbf{Here is an unordered list}:
\begin{itemize}
  \item
    An item

  \item
    Another item

\end{itemize}

\textbf{Here is an ordered list}:
\begin{enumerate}
  \item
    First item

  \item
    Second item

\end{enumerate}

\section*{Problem 2}

% YOUR SOLUTION GOES HERE

\section*{Problem 3}

% YOUR SOLUTION GOES HERE

\section*{Problem 4}

% YOUR SOLUTION GOES HERE

\section*{Problem 5}

% YOUR SOLUTION GOES HERE

\end{document}